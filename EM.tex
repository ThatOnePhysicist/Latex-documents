\documentclass[10pt,a4paper]{article}
\usepackage[margin=1in]{geometry}
\usepackage[utf8]{inputenc}
\usepackage{amsfonts}
\usepackage{amsmath}
\usepackage{amssymb}
\usepackage{amsthm}
\usepackage{cancel}
\usepackage{commath}
\usepackage{enumitem}
\usepackage{fancyhdr}
\usepackage{feynmp}
\usepackage{tikz}
\usepackage{siunitx}
\DeclareGraphicsRule{*}{mps}{*}{}
\usepackage{luatex85}
\usepackage{tikz-feynman}
\usepackage{graphicx}
\usepackage{mathtools}
\usepackage{ntheorem}
\usepackage{listings}
\usepackage{physics}
\usepackage{stackengine,graphicx}

\renewcommand\arraystretch{2.5}


\newcommand\frightarrow{\scalebox{1}[.4]{$\rightarrow$}}
\newcommand\darrow[1][]{\mathrel{\stackon[1pt]{\stackanchor[1pt]{\frightarrow}{\frightarrow}}{\scriptstyle#1}}}
\newcommand{\Lim}[1]{\raisebox{0.5ex}{\scalebox{0.8}{$\displaystyle \lim_{#1}\;$}}}
\graphicspath{ {./Real analysis/} }

\newcommand{\eq}[2]{\begin{equation} \label{eq:#1} #2 \end{equation}} %define equation macro

\theoremstyle{break}
\newtheorem{theorem}{Theorem}

\definecolor{codegreen}{rgb}{0,0.6,0}
\definecolor{codegray}{rgb}{0.5,0.5,0.5}
\definecolor{codepurple}{rgb}{0.58,0,0.82}
\definecolor{backcolour}{rgb}{0.95,0.95,0.92}


\lstdefinestyle{mystyle}{
 %   backgroundcolor=\color{backcolour},
    commentstyle=\color{codegreen},
    keywordstyle=\color{blue},
    numberstyle=\tiny\color{codegray},
    stringstyle=\color{codepurple},
    basicstyle=\footnotesize,
    breakatwhitespace=true,
    breaklines=true,
    captionpos=b,
    keepspaces=true,
    numbers=left,
    postbreak=\mbox{\textcolor{red}{$\hookrightarrow$}\space},
  %  numbersep=5pt,
    showspaces=false,
    showstringspaces=false,
    showtabs=false,
    tabsize=2
}

\tikzfeynmanset{compat=1.1.0}

%-----------------PAGE STYLE------------------------
\pagestyle{fancy}
\fancyhf{}
\rhead{}
\lhead{}
\rfoot{\thepage}
%----------------BEGIN------------------------------
\begin{document}
  \begin{enumerate}
%%%%%%%%%%%%%%%%%%%%%%%%%%%%%%%%%%%%%%%%%%%%%%%%%%%%%%%%%%%%%%%%%%%%%%%%%%%%%%%%%%%%%%%%%%%%%
%%%%%%%%%%%%%%%%%%%%%%%%%%%%%%%%%%QUESTION 1%%%%%%%%%%%%%%%%%%%%%%%%%%%%%%%%%%%%%%%%%%%%%%%%%
%%%%%%%%%%%%%%%%%%%%%%%%%%%%%%%%%%%%%%%%%%%%%%%%%%%%%%%%%%%%%%%%%%%%%%%%%%%%%%%%%%%%%%%%%%%%%
    \item Show that under the Lorentz transformation, $\tilde{u} \equiv d\textbf{\textit{x}}/dt$ transforms as
    \begin{equation}
      \tilde{u}'= \cfrac{d\textbf{\textit{x}}'}{dt} =
      \begin{pmatrix}
        c \cfrac{dt'}{dt'} \\
        \cfrac{d\Vec{x}'}{dt'}
      \end{pmatrix} =
      \begin{pmatrix}
        c \\
        \Vec{u}
      \end{pmatrix} =
      \begin{pmatrix}
        c \\
        \cfrac{u_x - v}{1 - \frac{vu_x}{c^2}}\\
        \cfrac{u_y - v}{\gamma\left(1 - \frac{vu_y}{c^2}\right)}\\
        \cfrac{u_z - v}{\gamma\left(1 - \frac{vu_z}{c^2}\right)}
      \end{pmatrix}
    \end{equation}
    where $\gamma \equiv 1/\sqrt{1-v^2/c^2}$. Note that the spatial components are 3-velocity addition formula! We see that $d\textbf{\textit{x}}/dt$ does not transform like a 4-vector under Lorentz transformation, despite it is a 4-component object.
    \newline\\
    \textbf{\textit{Solution}}:
    \newline\\
    Consider the following reletavistic relation, \[(t',x',y',z') = \left(\gamma(t-v/c^2x),\gamma(x-vt),y,z\right).\]
    Applying the differential operator yields,
    \[(dt',dx',dy',dz') = \left(\gamma(dt-v/c^2dx),\gamma(dx-vdt),dy,dz\right).\]
    Dividing each term by $dt'$ gives us the spatial component of $\widetilde{u}'$ with the $u^0$ component being $d(ct)/dt = c$,
    \[\widetilde{u}' = \begin{pmatrix}
      c\\
      \cfrac{\gamma\left(dx - vdt\right)}{\gamma \left(dt - \cfrac{v}{c^2}dx\right)}\\
      \cfrac{dy}{\gamma \left(dt - \cfrac{v}{c^2}dx\right)}\\
      \cfrac{dz}{\gamma \left(dt - \cfrac{v}{c^2}dx\right)}
    \end{pmatrix}
    = \begin{pmatrix}
      c\\
      \left(\cfrac{dx - vdt}{dt - \cfrac{v}{c^2}dx}\right)\left(\cfrac{\cfrac{1}{dt}}{\cfrac{1}{dt}}\right)\\
      \left(\cfrac{dy}{\gamma \left(dt - \cfrac{v}{c^2}dx\right)}\right)\left(\cfrac{\cfrac{1}{dt}}{\cfrac{1}{dt}}\right)\\
      \left(\cfrac{dz}{\gamma \left(dt - \cfrac{v}{c^2}dx\right)}\right)\left(\cfrac{\cfrac{1}{dt}}{\cfrac{1}{dt}}\right)
    \end{pmatrix}\\
    = \begin{pmatrix}
      c \\
      \cfrac{u_x - v}{1 - \frac{vu_x}{c^2}}\\
      \cfrac{u_y - v}{\gamma\left(1 - \frac{vu_y}{c^2}\right)}\\
      \cfrac{u_z - v}{\gamma\left(1 - \frac{vu_z}{c^2}\right)}
    \end{pmatrix}
    \]
%%%%%%%%%%%%%%%%%%%%%%%%%%%%%%%%%%%%%%%%%%%%%%%%%%%%%%%%%%%%%%%%%%%%%%%%%%%%%%%%%%%%%%%%%%%%%
%%%%%%%%%%%%%%%%%%%%%%%%%%%%%%%%%%QUESTION 2%%%%%%%%%%%%%%%%%%%%%%%%%%%%%%%%%%%%%%%%%%%%%%%%%
%%%%%%%%%%%%%%%%%%%%%%%%%%%%%%%%%%%%%%%%%%%%%%%%%%%%%%%%%%%%%%%%%%%%%%%%%%%%%%%%%%%%%%%%%%%%%
    \item Continue from the previous problem, where we found the transformation rule of the 4-component object $\tilde{u}$. Suppose we have another such 4-compnent object $\tilde{w}$. Calculate
      \begin{itemize}
        \item $\widetilde{u}' + \widetilde{w}'$, and
        \item $(\widetilde{u} + \widetilde{w})'$.
      \end{itemize}
    Do the two agree with each other?
    \newline\\
    \textbf{\textit{Solution}}:
    \newline\\
    From 1. we know $\widetilde{u}'$ and can easily infer $\widetilde{w}'$,
    \begin{equation*}
      \begin{split}
        \widetilde{u}' + \widetilde{w}' &= \begin{pmatrix}
          c \\
          \cfrac{u_x - v}{1 - \frac{vu_x}{c^2}}\\
          \cfrac{u_y - v}{\gamma\left(1 - \frac{vu_y}{c^2}\right)}\\
          \cfrac{u_z - v}{\gamma\left(1 - \frac{vu_z}{c^2}\right)}
        \end{pmatrix} +
        \begin{pmatrix}
          c \\
          \cfrac{w_x - v}{1 - \frac{vw_x}{c^2}}\\
          \cfrac{w_y - v}{\gamma\left(1 - \frac{vw_y}{c^2}\right)}\\
          \cfrac{w_z - v}{\gamma\left(1 - \frac{vw_z}{c^2}\right)}
        \end{pmatrix}\\
        &= \begin{pmatrix}
          2c \\
          \gamma_u^2(u_x-v) + \gamma^2_w(w_x-v)\\
          \frac{\gamma_u^2(u_y-v) + \gamma^2_w(w_y-v)}{\gamma}\\
          \frac{\gamma_u^2(u_z-v) + \gamma^2_w(w_z-v)}{\gamma}
        \end{pmatrix}\\
        &= \begin{pmatrix}
          2c \\
          \gamma_u^2u_x + \gamma_w^2w_x - v(\gamma^2_u+\gamma^2_w)\\
          \frac{\gamma_u^2u_y + \gamma_w^2w_y - v(\gamma^2_u+\gamma^2_w)}{\gamma}\\
          \frac{\gamma_u^2u_z + \gamma_w^2w_z - v(\gamma^2_u+\gamma^2_w)}{\gamma}
        \end{pmatrix}
      \end{split}
    \end{equation*}
    Let $\widetilde{r}' = (\widetilde{u}+\widetilde{w})'$
    \begin{equation*}
      \begin{split}
        \begin{pmatrix}
          c \\
          \cfrac{r_x - v}{1 - \frac{vr_x}{c^2}}\\
          \cfrac{r_y - v}{\gamma\left(1 - \frac{vr_y}{c^2}\right)}\\
          \cfrac{r_z - v}{\gamma\left(1 - \frac{vr_z}{c^2}\right)}
        \end{pmatrix}
      \end{split}
    \end{equation*}
    Since $(\widetilde{u}+\widetilde{w})'^0 \neq \widetilde{u}'^0 + \widetilde{w}'^0$, it is clear that $(\widetilde{u}+\widetilde{w})' \neq \widetilde{u}' + \widetilde{w}'$ and therefore don't agree with each other.
%%%%%%%%%%%%%%%%%%%%%%%%%%%%%%%%%%%%%%%%%%%%%%%%%%%%%%%%%%%%%%%%%%%%%%%%%%%%%%%%%%%%%%%%%%%%%
%%%%%%%%%%%%%%%%%%%%%%%%%%%%%%%%%%QUESTION 3%%%%%%%%%%%%%%%%%%%%%%%%%%%%%%%%%%%%%%%%%%%%%%%%%
%%%%%%%%%%%%%%%%%%%%%%%%%%%%%%%%%%%%%%%%%%%%%%%%%%%%%%%%%%%%%%%%%%%%%%%%%%%%%%%%%%%%%%%%%%%%%
    \item If a reference frame moves together with the particle, we say this reference frame is the \textit{comoving} reference frame of the particle. In other words, in the comoving frame the particle is stationary. Find the 4-velocity $\textbf{\textit{U}}$ of the particle in the comoving frame.
    \newline\\
    \textbf{\textit{Solution}}:
    \newline\\
    The 4-velocity $\textbf{\textit{U}}$ of a particle is the derivative of the 4-position with respect to proper time $\tau$,
    \[
    \textbf{\textit{U}} = \cfrac{dX^{\alpha}}{d\tau} =
      \begin{pmatrix}
        \cfrac{dx^0}{d\tau}\\
        \cfrac{dx^1}{d\tau}\\
        \cfrac{dx^2}{d\tau}\\
        \cfrac{dx^3}{d\tau}
      \end{pmatrix} =
      \begin{pmatrix}
        c\cfrac{dt}{d\tau}\\
        \cfrac{dx}{dt}\cfrac{dt}{d\tau}\\
        \cfrac{dy}{dt}\cfrac{dt}{d\tau}\\
        \cfrac{dz}{dt}\cfrac{dt}{d\tau}
      \end{pmatrix}=
      \begin{pmatrix}
        \gamma c\\
        \gamma u_x\\
        \gamma u_y\\
        \gamma u_z
      \end{pmatrix}=
      \begin{pmatrix}
        \gamma c \\
        \gamma \Vec{u}
      \end{pmatrix}
    \]
    Since we are in the comoving frame, $\Vec{u} = 0 \to \textbf{\textit{U}} = \begin{pmatrix} \gamma c \\ 0 \end{pmatrix}$

%%%%%%%%%%%%%%%%%%%%%%%%%%%%%%%%%%%%%%%%%%%%%%%%%%%%%%%%%%%%%%%%%%%%%%%%%%%%%%%%%%%%%%%%%%%%%
%%%%%%%%%%%%%%%%%%%%%%%%%%%%%%%%%%QUESTION 4%%%%%%%%%%%%%%%%%%%%%%%%%%%%%%%%%%%%%%%%%%%%%%%%%
%%%%%%%%%%%%%%%%%%%%%%%%%%%%%%%%%%%%%%%%%%%%%%%%%%%%%%%%%%%%%%%%%%%%%%%%%%%%%%%%%%%%%%%%%%%%%
    \item Show that for any particle traveling with a speed $v<c$, its 4-velocity statisfies $\textbf{\textit{U}}\cdot \textbf{\textit{U}} = -c^2$.
    \newline\\
    \textbf{\textit{Solution}}:
    \newline\\
    Consider being in the reference frame of some particle with 4-velocity $\textbf{\textit{U}} = \cfrac{dX^{\alpha}}{d\tau}$ where, $dt = \gamma{(v)}d\tau$ and $\gamma{(v)} \equiv \cfrac{1}{\sqrt{1-\frac{v^2}{c^2}}}$. But in the particle's reference frame, $\abs{v} = 0$ thus, $\gamma(v) = 1 \to dt = d\tau$
      \begin{equation*}
        %\begin{split}
          \textbf{\textit{U}} = \cfrac{dX^{\alpha}}{d\tau}
          &= \begin{pmatrix}
            \cfrac{dx^0}{d\tau} \\
            \cfrac{dx^1}{d\tau} \\
            \cfrac{dx^2}{d\tau} \\
            \cfrac{dx^3}{d\tau}
          \end{pmatrix}\\
          &= \begin{pmatrix}
            \cfrac{d(ct)}{d\tau} \\
            \cfrac{dx^1}{dt}\cfrac{dt}{d\tau} \\
            \cfrac{dx^2}{dt}\cfrac{dt}{d\tau} \\
            \cfrac{dx^3}{dt}\cfrac{dt}{d\tau}
          \end{pmatrix}\\
          &= \begin{pmatrix}
            c\\
            v^1\\
            v^2\\
            v^3
          \end{pmatrix}\\
          &= \begin{pmatrix}
            c\\
            0\\
            0\\
            0
          \end{pmatrix}\Rightarrow
          \textbf{\textit{U}} &= \begin{pmatrix}
            c \\
            \textbf{0}
        \end{pmatrix}
        %\end{split}
      \end{equation*}
      Assuming the Minkowski metric,
      \begin{equation*}
        \begin{split}
          \textbf{\textit{U}}\cdot \textbf{\textit{U}} &\equiv \sum \eta_{\alpha \beta}U^{\alpha}U^{\beta} \\
            &=
          \begin{pmatrix}
            -1 & 0 & 0 & 0\\
            0 & 1 & 0 & 0\\
            0 & 0 & 1 & 0 \\
            0 & 0 & 0 & 1
          \end{pmatrix}
          \begin{pmatrix}
            v^0 \\
            v^1 \\
            v^2 \\
            v^3
          \end{pmatrix}
          \begin{pmatrix}
            v^0 \\
            v^1 \\
            v^2 \\
            v^3
          \end{pmatrix}\\
          &= (-1)c^2 + \abs{\textbf{u}}^2\\
          &= \abs{\textbf{0}}^2 - c^2\\
          \therefore \textbf{\textit{U}}\cdot \textbf{\textit{U}} &= -c^2 \hspace{.5 cm} \square
        \end{split}
      \end{equation*}
%%%%%%%%%%%%%%%%%%%%%%%%%%%%%%%%%%%%%%%%%%%%%%%%%%%%%%%%%%%%%%%%%%%%%%%%%%%%%%%%%%%%%%%%%%%%%
%%%%%%%%%%%%%%%%%%%%%%%%%%%%%%%%%%QUESTION 5%%%%%%%%%%%%%%%%%%%%%%%%%%%%%%%%%%%%%%%%%%%%%%%%%
%%%%%%%%%%%%%%%%%%%%%%%%%%%%%%%%%%%%%%%%%%%%%%%%%%%%%%%%%%%%%%%%%%%%%%%%%%%%%%%%%%%%%%%%%%%%%
    \item For a particle of mass $m$ moving in the $\mathcal{S}$-frame whose spacetime coordinate is described by $\textbf{\textit{x}} = \left(ct, \Vec{x}(t)\right)$, its 4-momentum is defined as $\textbf{\textit{P}} \equiv m \textbf{\textit{U}}$, where \textbf{\textit{U}} is the particle's 4-veloctiy,
      \[\textbf{\textit{U}} = \frac{d\textbf{\textit{x}}}{d\tau}\]
    and $\tau$ is the proper time measured by the clock moving with the particle. With the relativistic energy $E$ and the reletavistic 3-momentum $\Vec{p}$ defined as the temporal and spatial components of the 4-momentumm as
      \[\textbf{\textit{P}} = \begin{pmatrix}
        E/c\\
        \Vec{p}
      \end{pmatrix},\]
    \textbf{show \textit{explicitly} that the relativistic energy and 3-momentum can be expressed as}
      \[E = \cfrac{mc^2}{\sqrt{1-v^2/c^2}}, \qquad \Vec{p} = \cfrac{m \Vec{v}}{\sqrt{1-v^2/c^2}},\]
    where $v \equiv \abs{\Vec{v}} = \abs{d\Vec{x}/dt}$.
    \newline\\
    \textbf{\textit{Solution}}:
    \newline\\
    $\textbf{\textit{P}}^{\alpha} = m \textbf{\textit{U}}^{\alpha}$ where, $\textbf{\textit{U}}$ is the 4-velocity and define $\gamma(v) = \frac{1}{\sqrt{1-v^2/c^2}}$ as usual.
    \[P^0 = mU^0 = \gamma mv^0\text{ and }P^i = mU^i = \gamma m v^i\]
    Consider the 0-component: given $P^0 = E/c$,
    \[\frac{E}{c} = \gamma mv^0 \to \frac{E}{c} = \gamma m c \to E = \gamma mc^2. \]
    Consider the $i$th-component where $i = 1,2, \text{ and }3$: given $P^i = \Vec{p}$ where $p^i$ is the momentum 3-vector,
    \[\Vec{p} = \gamma mv^i \to \Vec{p} = \gamma p^i\]
    Thus,
    \[E = \cfrac{mc^2}{\sqrt{1-v^2/c^2}}, \qquad \Vec{p} = \cfrac{m \Vec{v}}{\sqrt{1-v^2/c^2}}.\]
%%%%%%%%%%%%%%%%%%%%%%%%%%%%%%%%%%%%%%%%%%%%%%%%%%%%%%%%%%%%%%%%%%%%%%%%%%%%%%%%%%%%%%%%%%%%%
%%%%%%%%%%%%%%%%%%%%%%%%%%%%%%%%%%QUESTION 6%%%%%%%%%%%%%%%%%%%%%%%%%%%%%%%%%%%%%%%%%%%%%%%%%
%%%%%%%%%%%%%%%%%%%%%%%%%%%%%%%%%%%%%%%%%%%%%%%%%%%%%%%%%%%%%%%%%%%%%%%%%%%%%%%%%%%%%%%%%%%%%
    \item
      \begin{enumerate}
        \item[(a)] Show that the reletavistic energy $E$ and the 3-momentum $\Vec{p}$ satisfies the following identity
        \[E^2 - \abs{\Vec{p}}^2c^2 = m^2c^4\]
        \item[(b)] Express the speed $v$ of the particle in terms of the relativistic 3-momentum $\abs{\Vec{p}}$, relativistic energy $E$, and the speed of light $c$.
        \item[(c)] With the relations (a) and (b), what is the relation between $\abs{\Vec{p}}$ and $E$ for a massless particle and how fast should it travel?
      \end{enumerate}
      \newline\\
      \textbf{\textit{Solution}}:
      \newline\\
      (a) Assuming the Minkowski metric, $\eta = (-,+,+,+)$, consider from 5., $\textbf{\textit{P}} = \begin{pmatrix}
        E/c\\
        \Vec{p}
      \end{pmatrix}$.
      \begin{equation*}
          \begin{split}
              \textbf{\textit{P}}^2 &= \textbf{\textit{P}} \cdot \textbf{\textit{P}}\\
                &= \sum \eta_{\alpha \beta}P^{\alpha}P^{\beta}\\
                &= -\left(P^0\right)^2 + \left(P^i\right)^2\\
                &= -\left(\frac{E}{c}\right)^2 + \abs{\Vec{p}}^2
          \end{split}
      \end{equation*}
      From 4. we know that $\textbf{\textit{P}}^2 = -mc^2$ where $m$ is the rest mass of the object. Thus,
      \begin{equation*}
        \begin{split}
          -m^2c^2 &= -\left(\frac{E}{c}\right)^2 + \abs{\Vec{p}}^2\\
          &\text{ or }\\
          E^2 - \abs{\Vec{p}}^2c^2 &= m^2c^4
        \end{split}
      \end{equation*}
      (b) From 5, $\Vec{p} = \gamma m \Vec{v}$ and $E = \gamma mc^2$
      \begin{equation*}
            \begin{split}
              E^2 - \abs{\Vec{p}}^2c^2 &= m^2c^4\\
              E^2 - \abs{\gamma m\Vec{v}}^2c^2 &= m^2c^4\\
              E^2 + m^2c^4 &= \abs{\gamma m\Vec{v}}^2c^2\\
              v &= \sqrt{\frac{E^2 + m^2c^4}{\left(\gamma mc\right)^2}}\\
              v &= \frac{\sqrt{\left(\gamma^2 + 1\right)m^2c^4}}{\gamma mc}\\
              v &= \frac{mc^2\sqrt{\gamma^2+1}}{\gamma mc}\\
              v &= \frac{c\sqrt{\gamma^2 + 1}}{\gamma}
            \end{split}
          \end{equation*}
      (c)
%%%%%%%%%%%%%%%%%%%%%%%%%%%%%%%%%%%%%%%%%%%%%%%%%%%%%%%%%%%%%%%%%%%%%%%%%%%%%%%%%%%%%%%%%%%%%
%%%%%%%%%%%%%%%%%%%%%%%%%%%%%%%%%%QUESTION 7%%%%%%%%%%%%%%%%%%%%%%%%%%%%%%%%%%%%%%%%%%%%%%%%%
%%%%%%%%%%%%%%%%%%%%%%%%%%%%%%%%%%%%%%%%%%%%%%%%%%%%%%%%%%%%%%%%%%%%%%%%%%%%%%%%%%%%%%%%%%%%%
    \item Show that when a particle's speed $\abs{\Vec{v}}$ is much smaller than $c$,
      \begin{enumerate}
        \item[(i.)] the reletavistic 3-momentum is reduced to the momentum in Newtonian mechanics.
        \item[(ii.)] the relativistic energy is reduced to a rest mass energy $E_0 = mc^2$ plus the kinetic energy defined in Newtonian mechanics.
      \end{enumerate}
      \newline\\
      \textbf{\textit{Solution}}:
      \newline\\
      (i.) For $0<v\ll c \to 0<\left(\cfrac{v}{c}\right)^2 \ll 1$, therefore,
          \[\Vec{p} = \frac{m\Vec{v}}{\sqrt{1- \frac{v^2}{c^2}}} \approx m\Vec{v}\]
      (ii.) For $0<v\ll c \to \left(\cfrac{v}{c}\right)^2 \ll 1$, therefore,
          \begin{equation*}
            \begin{split}
              E^2 &= m^2c^4 + \abs{\Vec{p}}^2c^2\\
              E &= \sqrt{m^2c^4 + \abs{\Vec{p}}^2c^2}\\
              &= \sqrt{m^2c^4 + m^2v^2c^2}\\
              &= \sqrt{m^2c^2(c^2+v^2)}\\
              &= mc\sqrt{c^2+v^2}\\
              E &\approx mc^2
            \end{split}
          \end{equation*}
%%%%%%%%%%%%%%%%%%%%%%%%%%%%%%%%%%%%%%%%%%%%%%%%%%%%%%%%%%%%%%%%%%%%%%%%%%%%%%%%%%%%%%%%%%%%%
%%%%%%%%%%%%%%%%%%%%%%%%%%%%%%%%%%QUESTION 8%%%%%%%%%%%%%%%%%%%%%%%%%%%%%%%%%%%%%%%%%%%%%%%%%
%%%%%%%%%%%%%%%%%%%%%%%%%%%%%%%%%%%%%%%%%%%%%%%%%%%%%%%%%%%%%%%%%%%%%%%%%%%%%%%%%%%%%%%%%%%%%
    \item Consider a $2 \to 2$ particle scattering, where $A$ and $B$ collide and come out with particle $C$ and $D$. Suppose in the $\mathcal{S}$-frame the 4-momentum is conserved, i.e.
      \[\textbf{\textit{P}}_1 + \textbf{\textit{P}}_2 = \textbf{\textit{P}}_3 + \textbf{\textit{P}}_4.\]
    Show that if the 4-momentums $\textbf{\textit{P}}_i$ transform linearly under Lorentz transformations, that is
      \[\Lambda(a\textbf{\textit{P}}_i + b \textbf{\textit{P}}_j) = a \Lambda(\textbf{\textit{P}}_i)+b \Lambda(\textbf{\textit{P}}_j)\]
    for any real numbers $a$ and $b$ and 4-momentums $\textbf{\textit{P}}_i$ and $\textbf{\textit{P}}_j$, then the 4-momentum conservation also holds true in other inertial reference frams, i.e.
      \[\textbf{\textit{P}}_1' + \textbf{\textit{P}}_2' = \textbf{\textit{P}}_3' + \textbf{\textit{P}}_4'.\]
      \newline\\
      \textbf{\textit{Solution}}:
      \newline\\
      Given $\textbf{\textit{P}}_1 + \textbf{\textit{P}}_2 = \textbf{\textit{P}}_3 + \textbf{\textit{P}}_4$ with constants $a=b=1$ and their linearity under Lorentz transformations,
      \begin{equation*}
        \begin{split}
          \Lambda \left(\textbf{\textit{P}}_1 + \textbf{\textit{P}}_2\right) &= \Lambda \left(\textbf{\textit{P}}_3 + \textbf{\textit{P}}_4\right)\\
          \Lambda \left(\textbf{\textit{P}}_1\right) + \Lambda \left(\textbf{\textit{P}}_2\right) &= \Lambda \left(\textbf{\textit{P}}_3\right) + \Lambda \left(\textbf{\textit{P}}_4\right)\\
          \textbf{\textit{P}}_1' + \textbf{\textit{P}}_2' &= \textbf{\textit{P}}_3' + \textbf{\textit{P}}_4'.
        \end{split}
      \end{equation*}
    \end{enumerate}
\end{document}
