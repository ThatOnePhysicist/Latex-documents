\documentclass[10pt,a4paper]{article}
\usepackage[margin=1in]{geometry}
\usepackage[utf8]{inputenc}
\usepackage{amsfonts}
\usepackage{amsmath}
\usepackage{amssymb}
\usepackage{amsthm}
\usepackage{cancel}
\usepackage{commath}
\usepackage{enumitem}
\usepackage{fancyhdr}
\usepackage{feynmp}
\usepackage{tikz}
\usepackage{siunitx}
\DeclareGraphicsRule{*}{mps}{*}{}
\usepackage{luatex85}
\usepackage{tikz-feynman}
\usepackage{graphicx}
\usepackage{mathtools}
\usepackage{ntheorem}
\usepackage{listings}
\usepackage{physics}
\usepackage{stackengine,graphicx}

\newcommand\frightarrow{\scalebox{1}[.4]{$\rightarrow$}}
\newcommand\darrow[1][]{\mathrel{\stackon[1pt]{\stackanchor[1pt]{\frightarrow}{\frightarrow}}{\scriptstyle#1}}}
\newcommand{\Lim}[1]{\raisebox{0.5ex}{\scalebox{0.8}{$\displaystyle \lim_{#1}\;$}}}
\graphicspath{ {./Real analysis/} }

\newcommand{\eq}[2]{\begin{equation} \label{eq:#1} #2 \end{equation}} %define equation macro

\theoremstyle{break}
\newtheorem{theorem}{Theorem}

\definecolor{codegreen}{rgb}{0,0.6,0}
\definecolor{codegray}{rgb}{0.5,0.5,0.5}
\definecolor{codepurple}{rgb}{0.58,0,0.82}
\definecolor{backcolour}{rgb}{0.95,0.95,0.92}

\lstdefinestyle{mystyle}{
 %   backgroundcolor=\color{backcolour},
    commentstyle=\color{codegreen},
    keywordstyle=\color{blue},
    numberstyle=\tiny\color{codegray},
    stringstyle=\color{codepurple},
    basicstyle=\footnotesize,
    breakatwhitespace=true,
    breaklines=true,
    captionpos=b,
    keepspaces=true,
    numbers=left,
    postbreak=\mbox{\textcolor{red}{$\hookrightarrow$}\space},
  %  numbersep=5pt,
    showspaces=false,
    showstringspaces=false,
    showtabs=false,
    tabsize=2
}

\tikzfeynmanset{compat=1.1.0}

%-----------------PAGE STYLE------------------------
\pagestyle{fancy}
\fancyhf{}
\rhead{}
\lhead{}
\rfoot{\thepage}
%----------------BEGIN------------------------------
\begin{document}
  \section*{8 Conservation Laws}
    \subsubsection*{8.1 Charge and energy}
      \subsubsection*{8.1.1 The continuity equation}
        Conservation of charge: \newline
          The charge in a volume $\mathcal{V}$ is
            \begin{equation}\tag{8.1}
              Q(t) = \int_{\mathcal{V}} \rho(\textbf{r},t)d\tau
            \end{equation}
          and the current flowing through the boundary $\mathcal{S}$ is $\oint_{\mathcal{S}}\textbf{J}\cdot d\mathbf{a}$ so local charge conservation says
            \begin{equation}\tag{8.2}
              \frac{dQ}{dt} = -\oint_{\mathcal{S}}\mathbf{J}\cdot d \mathbf{a}
            \end{equation}
          Applying divergence theorem to $(8.1)$
            \begin{equation}\tag{8.3}
              \int_{\mathcal{V}}\pd\rho td\tau = - \int_{\mathcal{V}}\nabla \cdot \mathbf{J}d\tau
            \end{equation}
          since this is true for any volume, the continuity equation:
            \begin{equation}\tag{8.4}
              \boxed{\pd \rho t = -\nabla \cdot \mathbf{J}}
            \end{equation}
      \subsubsection*{8.1.2 Poynting's thorem}
        Work needed to create a static charge distribution,
          \begin{equation*}
            W_e = \frac{\epsilon_0}{2}\int E^2 d\tau.
          \end{equation*}
        Work needed to get currents going,
          \begin{equation*}
            W_m = \frac{1}{2\mu_0}\int B^2 d\tau.
          \end{equation*}
        The total energy stored in the electromagnetic fields,
          \begin{equation}\tag{8.5}
            \boxed{u = \frac{1}{2}\left(\epsilon_0E^2 + \frac{1}{\mu_0}B^2\right).}
          \end{equation}
        Suppose we have some charge and current configuration which, at time $t$, produces fields \textbf{E} and \textbf{B}. In the next instant, $dt$, the chages move aorund a bit. Using the Lorentz force law, the work done on a charge q is
          \begin{equation*}
            \textbf{F} \cdot d\textbf{l} = q\left(\textbf{E} + \textbf{v}\cross\textbf{B}  \right)\cdot \textbf{v} dt = q\textbf{E}\cdot \textbf{v}dt.
          \end{equation*}
        In terms of the charge and current densities, $q \to \rho d\tau$ and $\rho\textbf{v}\to \textbf{J}^2$, the rate at which work is done on all the charges in a volume $\mathcal{V}$ is
          \[\frac{dW}{dt} = \int_{\mathcal{V}}\textbf{E}\cdot\textbf{J}d\tau  \]
\end{document}
